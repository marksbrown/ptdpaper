The uncertainty in arrival time of a particle striking a scintillator detector is non-zero. 
%TODO insert definition of the CTR


...the time resolution, has several contributions. These broadly break into three groups; the intrinsic properties of the scintillator crystal, transit time jitter %TODO use Stefan's terminology and acroynms

(TTS) within the photodetector and timing jitter due to the electronics used. In this paper we consider a fourth contribution quantitatively; namely the light transport within the scintillator crystal. 
%TODO define light transport.

An improvement in any of the above contributions will lead to an improvement in the overall timing performance.

In recent years the discovery of new scintillator materials, such as LSO, with high scintillation yield and low decay rates has led to a better time resolution \cite{Dorenbos2010a,Conti2009a}. 

At these finer resolutions smaller effects, such as the intrinsic rise time of the material, will cause a noticeable degradation. Greater understanding of these contributions will allow future scintillator detectors to be designed to minimise these effects. 

To incorporate time-of-flight (TOF) information into the next generation of Positron-Emission Tomography (PET), the time resolution of scintillator detectors must be improved. Furthermore this must be done without compromising other factors such as sensitivity and spatial resolution \cite{Eriksson2004}. Incorporation of TOF information within PET will reduce noise, improving the quality of PET scanner images \cite{Geramifar2011}.

In this paper we calculate the contribution of light transport to the time resolution using the Cram\'{e}r-Rao lower bound. The Cram\'{e}r-Rao lower bound, also known as the information inequality, gives the lowest potential variance in the estimator of a parameter. The parameter in this case is the arrival time of a
gamma ray photon in the scintillator detector. We expand upon prior work \cite{Seifert2012} by direct inclusion of Monte-Carlo simulation results. By doing so a powerful insight into the contribution of various properties, such as scintillator crystal length, to the time resolution is ascertained. It is shown in this paper that the contribution to the time resolution due to light transport, when scintillator crystal wrapping is considered, is substantial.